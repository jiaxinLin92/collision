% What kind of text document should we build
\documentclass[a4,10pt]{article}


% Include packages we need for different features (the less, the better)

% Clever cross-referencing
\usepackage{cleveref}

% Math
\usepackage{amsmath}

% Algorithms
\usepackage{algorithm}
\usepackage{algpseudocode}
\usepackage{titlesec}
\usepackage{lipsum}% just to generate text for the example
\usepackage{listings} 
\usepackage{url}

\titleformat{\section}%
  [hang]% <shape>
  {\normalfont\bfseries\Large}% <format>
  {}% <label>
  {0pt}% <sep>
  {}% <before code>
\renewcommand{\thesection}{}% Remove section references...
\renewcommand{\thesubsection}{\arabic{subsection}}

% Tikz
\RequirePackage{tikz}
\usetikzlibrary{arrows,shapes,calc,through,intersections,decorations.markings,positioning}

\tikzstyle{every picture}+=[remember picture]

\RequirePackage{pgfplots}




% Set TITLE, AUTHOR and DATE
\title{Advanced Game- and Simulator Programming(STE6245 )}
\author{Jiaxin Lin(140740)}
\date{\today}
 


\begin{document}



  % Create the main title section
  \maketitle

  \begin{abstract}
   This is the report about the Simulation of collision.Using GMlib, class Sphere and class Psurf from UiT-Campus Narvik.The report main to
description of GMlib,programming style,the important part of collision detection system
 and demo of project.
   
  \end{abstract}


  %%%%%%%%%%%%%%%%%%%%%%%%%%%%%%%%%%%%%%
  %%  The main content of the report  %%
  %%%%%%%%%%%%%%%%%%%%%%%%%%%%%%%%%%%%%%
 
  \section{1 Introduction}
This is a description of GMlib Library,the main class use in the project and the C++ programming style. 
%\vspace{10pt}
 \subsection*{1.1 GMlib Library}
 %\vspace{10pt}
GMlib is free software for use with geometric modelling with parametric in C++.In the project,
Visualization of the balls and walls are using GMlib.The libary was make and update by UIT-Campus Narvik.

   \subsection*{1.2 Main Class}
In this project I creat class Testsphere,Controller inherit from PSphere, class MYPlanenine inherit form PSurf,class MP inherit from PPlane.The PSphere and PPlane both inherit from Psurf.
%\vspace{10pt}
   \subsubsection*{1.2.1 PSphere Class}
This class to creat balls.In this class have some functions.
\begin{table}[H]
\centering
\caption{Class PSphere main functions}
\label{my-label}
\begin{tabular}{lll}
\cline{1-2}
\multicolumn{1}{|l|}{\textbf{Functions}} & \multicolumn{1}{l|}{\textbf{Means}}                                                                                                                                                                    &  \\ \cline{1-2}
\multicolumn{1}{|l|}{PSphere}            & \multicolumn{1}{l|}{\begin{tabular}[c]{@{}l@{}}This is the construction function to create the ball.\\ And have overload the constructor.\end{tabular}}                                                &  \\ \cline{1-2}
\multicolumn{1}{|l|}{getRadius}          & \multicolumn{1}{l|}{This function to get the sphere radius.}                                                                                                                                           &  \\ \cline{1-2}
\multicolumn{1}{|l|}{setRadius}          & \multicolumn{1}{l|}{This function to set the sphere radius.}                                                                                                                                           &  \\ \cline{1-2}
\multicolumn{1}{|l|}{eval}               & \multicolumn{1}{l|}{\begin{tabular}[c]{@{}l@{}}This function to set matrix.Using formula to \\ createthe shape like the ball in 3D.\\ (the function parameter member inherit from PSurf)\end{tabular}} &  \\ \cline{1-2}

\end{tabular}
\end{table}
\subsubsection*{1.2.2 PPlane Class}
This class to creat walls.The main function follow in the Table 2.

\begin{table}[H]
\centering
\caption{Class PPlane main functions}
\label{my-label}
\begin{tabular}{ll}
\hline
\multicolumn{1}{|l|}{\textbf{Functions}} & \multicolumn{1}{l|}{\textbf{Means}}                                                                                                                   \\ \hline
\multicolumn{1}{|l|}{PPlane}             & \multicolumn{1}{l|}{\begin{tabular}[c]{@{}l@{}}This is the construction function to create the plane.\\ Need one point and two vectors.\end{tabular}} \\ \hline
\multicolumn{1}{|l|}{getNormal}          & \multicolumn{1}{l|}{This function to get the normal of the surfaces.}                                                                                 \\ \hline
\multicolumn{1}{|l|}{getU}               & \multicolumn{1}{l|}{This function to get the Vector U(vector u use to create wall).}                                                                  \\ \hline
\multicolumn{1}{|l|}{getV}               & \multicolumn{1}{l|}{This function to get the Vector V(vector v use to create wall).}                                                                  \\ \hline
\multicolumn{1}{|l|}{setU}               & \multicolumn{1}{l|}{This function to set the Vector U(vector u use to create wall).}                                                                  \\ \hline
\multicolumn{1}{|l|}{setV}               & \multicolumn{1}{l|}{This function to set the Vector V(vector v use to create wall).}                                                                  \\ \hline
\multicolumn{1}{|l|}{setP}               & \multicolumn{1}{l|}{This function to set the point of create the wall.}                                                                               \\ \hline
                                                                                                                                                                                            
\end{tabular}
\end{table}

\subsubsection*{1.2.3 PSurf Class}
This is the basic class to create the surface.This class inherit from Parametrics.
The class PSurf have lots of functions which use to creat the surface you want in 3D.
Just look the inheritance diagram follow the figure.

    \begin{figure}[H]
      \centering
      \includegraphics[width=1.00\textwidth]{gfx/Psurf.png}
      \caption{Inheritance diagram for Psurf}
      \label{fig:psurf}
    \end{figure}

   \subsection*{1.3 Fundamental Types}

\begin{table}[H]
\centering
\caption{ Main Types}
\label{my-label}
\begin{tabular}{ll}
\hline
\multicolumn{1}{|l|}{\textbf{Fundamental Types}} & \multicolumn{1}{l|}{\textbf{Means}}                                                                                                                                                  \\ \hline
\multicolumn{1}{|l|}{GMlib::Vector}              & \multicolumn{1}{l|}{\begin{tabular}[c]{@{}l@{}}In the project this is the important type of variable.\\ The vector have length and direction.It is a template\\ class.\end{tabular}} \\ \hline
\multicolumn{1}{|l|}{GMlib::Point}               & \multicolumn{1}{l|}{\begin{tabular}[c]{@{}l@{}}This type to create a point.It is a template like the \\ vector.\end{tabular}}                                                        \\ \hline
\multicolumn{1}{|l|}{GMlib::Array}               & \multicolumn{1}{l|}{\begin{tabular}[c]{@{}l@{}}This type to create a array.It is a container to store \\ the number of balls,walls and collision.\end{tabular}}                      \\ \hline
                                                                                                                                                                                     
\end{tabular}
\end{table}

   \subsection*{1.4 Programming Style}
In this project follow some programming style.
 \begin{itemize}
\item Using Allman style.if, for, while, namespace braces, write in a new line.
\item Naming class using uppercase.
\item  Private member variables using underline.
\item Naming the type bool add is(such as isEmpty).
\item Using tab to indent.
\end{itemize}


  \section{2 QT Framework}

Qt is used mainly for developing application software with graphical user interfaces.
 Qt supports many compilers, including the GCC C++ compiler and the Visual Studio suite.
Qt 5 was officially released on 19 December 2012. This new version marked a major change in the platform, with hardware-accelerated graphics, QML and JavaScript playing a major role. The traditional C++-only QWidgets continued to be supported, but did not benefit from the performance improvements available through the new architecture.\cite{IEEEhowto:two} Qt 5 brings significant improvements to the speed and ease of developing user interfaces.\cite{IEEEhowto:three}In this project using Qt 
Version 5.4.


  \section{3 Theory}
This part introduce some theory whic use in this project.First give some concept and then description of class.
   \subsection*{3.1 Concept-Compute Step}
    This concepet use to calculate the ball next step when the ball move.
    \begin{figure}[H]
      \centering
      \includegraphics[width=1.00\textwidth]{gfx/ComputeStep.png}


      \caption{The ball move}
      \label{fig:fi}
    \end{figure}

According the project 

\begin{flalign}
\begin{split}
 \_{ ds}=dt*\_{velocity} + (0.5*dt*dt)*g
\end{split}&
\end{flalign}

\begin{flalign}
\begin{split}
 \_{p} = P*ds
\end{split}&
\end{flalign}

\begin{flalign}
\begin{split}
 \_n ={m[0][0]} + r*n
\end{split}&
\end{flalign}

\begin{flalign}
\begin{split}
 \_{ ds}=n-P
\end{split}&
\end{flalign}

   \subsection*{3.2 Concept-Ball Wall Collision}
	This is the main part of collision.First need to know where is the wall.
Creat wall need parameter follow the table.
\begin{table}[H]
\centering
\caption{My caption}
\label{my-label}
\begin{tabular}{ll}
\hline
\multicolumn{1}{|l|}{\textbf{Variable}} & \multicolumn{1}{l|}{\textbf{Means}}                         \\ \hline
\multicolumn{1}{|l|}{Vector1}           & \multicolumn{1}{l|}{one direction of create the wall}       \\ \hline
\multicolumn{1}{|l|}{Vector2}           & \multicolumn{1}{l|}{other direction of create the wall}     \\ \hline
\multicolumn{1}{|l|}{Point}             & \multicolumn{1}{l|}{Decide the position to create the wall} \\ \hline                                        
\end{tabular}
\end{table}

    \begin{figure}[H]
      \centering
      \includegraphics[width=1.10\textwidth]{gfx/BW.png}
      \caption{Ball Wall Collision}
      \label{fig:fi}
    \end{figure}

From this figure can see clearly how to detect the ball and wall collision and caculate the related data.

\begin{flalign}
\begin{split}\label{eq:some}
<d+xds,n>=r 
\end{split}&
from\quad the\quad red\quad  line\quad  .
\end{flalign}
Solve this formular to caculate x.

\begin{flalign}
\begin{split}\label{eq:some}
<d,n>+x<ds,n>=r 
\end{split}&
\end{flalign}

\begin{flalign}
\begin{split}\label{eq:some}
x=\frac{r-<d,n>}{<ds,n>}
\end{split}&
\end{flalign}
Check if the value of ds*n less than 0 and the variable x between 0 and 1 the collision is detected.

\begin{flalign}
\begin{split}\label{eq:some}
V = V -2*(n*V)*n
\end{split}&
\end{flalign}
Set new velocity after collision.

   \subsection*{3.3 Concept-Ball Ball Collision}
 This part of ball with ball collision.The ball constructor
have some parameter follow the table.

\begin{table}[H]
\centering
\caption{ball constructor parameter}
\label{my-label}
\begin{tabular}{ll}
\hline
\multicolumn{1}{|l|}{\textbf{Parameter}} & \multicolumn{1}{l|}{\textbf{Means}}                                            \\ \hline
\multicolumn{1}{|l|}{mass}               & \multicolumn{1}{l|}{The mass of the ball.}                                     \\ \hline
\multicolumn{1}{|l|}{velocity}           & \multicolumn{1}{l|}{The velocity of the ball}                                  \\ \hline
\multicolumn{1}{|l|}{radius}             & \multicolumn{1}{l|}{From the GMlib::PSphere,the radius of the ball}            \\ \hline
\multicolumn{1}{|l|}{plane}              & \multicolumn{1}{l|}{The ball move on the plane.Decide what kind of the plane.} \\ \hline
                                                                         
\end{tabular}
\end{table}


    \begin{figure}[H]
      \centering
      \includegraphics[width=1.10\textwidth]{gfx/BB.png}
      \caption{Ball Ball Collision}
      \label{fig:fi}
    \end{figure}

From the figure can make an equation.

\begin{flalign}
\begin{split}\label{eq:some}
P_{1}+xds1-(p_{2}+xds2)=p_{1}-p_{2}+x(ds1-ds2)=dp+xds
\end{split}&
\end{flalign}

\begin{flalign}
\begin{split}\label{eq:some}
<dp+xds,dp+xds>=r^{2}
\end{split}&
\end{flalign}

Simplification
\begin{flalign}
\begin{split}\label{eq:some}
<ds,ds>x^{2}+2<ds,dp>x+<dp,dp>-r^{2}=0
\end{split}&
\end{flalign}

To caculate x,x should between 0 and 1.\\
\\
$a=<ds,ds>$ \\
$b=2<ds,dp>x$\\
$c=<dp,dp>-r^{2}$

\begin{flalign}
\begin{split}\label{eq:some}
x=\frac{-b\pm{\sqrt{b^{2}-4ac}}}{2a}
\end{split}&
\end{flalign}

The new velocity after collision follow the below figure.


    \begin{figure}[H]
      \centering
      \includegraphics[width=1.20\textwidth]{gfx/BBV.png}
      \caption{set new velocity}
      \label{fig:fi}
    \end{figure}

From the figure can get some formula.
\begin{flalign}
\begin{split}\label{eq:some}
d=p2-p1
\end{split}&
\end{flalign}
The ball 1
\begin{flalign}
\begin{split}
v1=(V1*d)*d \quad\quad
v1n=V1-v1
\end{split}&
\end{flalign}
The ball 2
\begin{flalign}
\begin{split}
v2=(V2*d)*d \quad\quad
v2n=V2-v2
\end{split}&
\end{flalign}
The new velocity with mass
\begin{flalign}
\begin{split}
V1after=\frac{m1-m2}{m1+m2}*v1+\frac{2*m1}{m1+m2}*v2
\end{split}&
\end{flalign}

\begin{flalign}
\begin{split}
V2after=\frac{m2-m1}{m1+m2}*v1+\frac{2*m2}{m1+m2}*v2
\end{split}&
\end{flalign}
The finnal velocity of the ball
\begin{flalign}
\begin{split}
V1new=V1after+v1n
\end{split}&
\end{flalign}

\begin{flalign}
\begin{split}
V2new=V2after+v2n
\end{split}&
\end{flalign}

  \section{4 Structure of the resulating application}
\subsection*{4.1 Structure of this project}
The project base on teacher's framework.
In this project add five parts,the main part is the Controller. The planenine means 
using nine point to create floor.
    \begin{figure}[H]
      \centering
      \includegraphics[width=1.10\textwidth]{gfx/structure.png}
      \caption{The structure of the project}
      \label{fig:fi}
    \end{figure}

\subsection*{4.2 Description of each class}
Class list.Here are the classes.

\begin{table}[H]
\centering
\caption{Class List}
\label{my-label}
\begin{tabular}{ll}
\hline
\multicolumn{1}{|l|}{\textbf{Classes}} & \multicolumn{1}{l|}{\textbf{Brief descriptions}}                                                                                      \\ \hline
\multicolumn{1}{|l|}{Testsphere}       & \multicolumn{1}{l|}{Inherit the PSphere.To create a ball.}                                                                            \\ \hline
\multicolumn{1}{|l|}{MP}               & \multicolumn{1}{l|}{Inherit the PPlane.In this project to create walls.}                                                              \\ \hline
\multicolumn{1}{|l|}{MYPlanenine}      & \multicolumn{1}{l|}{Inherit the PSurf.In this project to create the floor.}                                                           \\ \hline
\multicolumn{1}{|l|}{Collison}         & \multicolumn{1}{l|}{\begin{tabular}[c]{@{}l@{}}This class to define collision.Ball with ball and Ball\\ with wall\end{tabular}}       \\ \hline
\multicolumn{1}{|l|}{Controller}       & \multicolumn{1}{l|}{\begin{tabular}[c]{@{}l@{}}This class to find when have collision.The main part of \\ this project.\end{tabular}} \\
\hline 
\end{tabular}
\end{table}


\subsubsection*{4.2.1 Class Testsphere}
This class to define the ball.The properties of the ball included


\begin{table}[H]
\centering
\caption{Class Testsphere}
\label{my-label}
\begin{tabular}{ll}
\hline
\multicolumn{1}{|l|}{\textbf{Properties}}        & \multicolumn{1}{l|}{\textbf{Descriptions}}                                   \\ \hline
\multicolumn{1}{|l|}{GMlib::Vector \_velocity}   & \multicolumn{1}{l|}{The velocity of the ball.}                               \\ \hline
\multicolumn{1}{|l|}{int \_mass}                 & \multicolumn{1}{l|}{The mass of the ball.}                                   \\ \hline
\multicolumn{1}{|l|}{GMlib::Vector \_ds}         & \multicolumn{1}{l|}{The distance of the ball move.}                          \\ \hline
\multicolumn{1}{|l|}{MYPlanenine* \_plane}       & \multicolumn{1}{l|}{The floor of the ball move.Make connect with the floor.} \\ \hline
\multicolumn{1}{|l|}{double \_x}                 & \multicolumn{1}{l|}{The time to store.}                                      \\ \hline

\end{tabular}
\end{table}

\begin{table}[]
\centering
\caption{Functions of the Testsphere}
\label{my-label}
\begin{tabular}{|l|l|}
\hline
\textbf{Function} & \textbf{Means}                                \\ \hline
TestSphere        & Constructor,To set initial value of the ball. \\ \hline
setVelocity       & Set new velocity of the ball.                 \\ \hline
getVelocity       & Get velocity of the ball.                     \\ \hline
setMass           & Set new mass of the ball.                     \\ \hline
getMass           & Get mass of the ball.                         \\ \hline
setX              & Set run time of the ball.                     \\ \hline
getX              & Get the store of time.                        \\ \hline
moveUp            & Control the ball to move up.                  \\ \hline
moveDown          & Control the ball to move down.                \\ \hline
moveLeft          & Control the ball to move left.                \\ \hline
moveRight         & Control the ball to move right.               \\ \hline
computeStep       & Compute the ball move to next step.           \\ \hline
getSurfnormal     & To get the normal of the surface.             \\ \hline
\end{tabular}
\end{table}

\subsubsection*{4.2.2 Class MP}
This class inherit teacher's class PPlane.This class using to create walls.
Using one point and two vector.

\subsubsection*{4.2.3 Class MYPlanenine}
This class inherit teacher's class PSurf.This class using to create floors.
Using nine points set a matrix.

\subsubsection*{4.2.4 Class Collision}

\begin{table}[H]
\centering
\caption{Class Collision}
\label{my-label}
\begin{tabular}{|l|l|}
\hline
\textbf{Properties}         & \textbf{Descriptions}                          \\ \hline
Testsphere *\_sphere{[}2{] }            & The array of Testsphere class.                 \\ \hline
MP *\_wall;                 & Define the wall.                               \\ \hline
bool \_isSw;                & Define weather ball and wall collision or not. \\ \hline
double \_x                  & The time to store.                             \\ \hline
operator ==                 & Operator overloading to compare to collision.  \\ \hline
bool operator \textless     & Operator overloading to compare to collision.  \\ \hline
\end{tabular}
\end{table}

\subsubsection*{4.2.5 Class Controller}
\begin{table}[H]
\centering
\caption{ Class Controller}
\label{my-label}
\begin{tabular}{|l|l|}
\hline
\textbf{Properties}           & \textbf{Descriptions}                                                 \\ \hline
MYPlanenine* \_plane          & Define the floor surfaces.                                            \\ \hline
GMlib::Array \_sphereArray    & This is the array to contain balls.                                   \\ \hline
GMlib::Array \_wallArray      & This is the array to contain walls.                                   \\ \hline
GMlib::Array \_collisionArray & This is the array to contain collisions.                              \\ \hline
void addSphere                & Add ball into the ball array.                                         \\ \hline
void addWall                  & Add wall into the wall array.                                         \\ \hline
void findcollisionSW          & This is a function to find ball and wall collision.                   \\ \hline
void findcollisionSS          & This is a function to find ball and ball collision.                   \\ \hline
void collisionSW              & This is a function to set new velocity after ball and wall collision. \\ \hline
void collisionSS              & This is a function to set new velocity after ball and wall collision. \\ \hline
\end{tabular}
\end{table}

  \section{QT Keyboard control}
This part trying to control ball by keyboard.In the GMlibWrapper::keyPressed to set keys.
Using up,down,left,right to control ball velocity.

 \section{demo}
This part to show some interface in the project.
    \begin{figure}[H]
      \centering
      \includegraphics[width=1.00\textwidth]{gfx/demo1.png}
      \caption{demo 1}
      \label{fig:fi}
    \end{figure}

    \begin{figure}[H]
      \centering
      \includegraphics[width=1.00\textwidth]{gfx/demo2.png}
      \caption{demo 2}
      \label{fig:fi}
    \end{figure}


    \begin{figure}[H]
      \centering
      \includegraphics[width=1.00\textwidth]{gfx/demo3.png}
      \caption{demo 2}
      \label{fig:fi}
    \end{figure}


  \section{Conclusion}
    In conclusion,this project is  based on the lectures in the course: geometric modelling.
Througe this project understand the vector better.Practise better than theory.Moreover,with use the 
GMlib libary and see the code form the teacher.
Final need to say thank you to my classmates.They help me a lot.
  
% Include the bibliography
\begin{thebibliography}{1}


%\bibitem{h}
%\url{http://episteme.hin.no/dox/GMlib2/classGMlib_1_1TriangleFacets.html}


\bibitem{IEEEhowto:two}
Qt5-feedback (Mailing list), \emph{Concern about removal of QWidget classes},\hskip 1em plus
  0.5em minus 0.4em\relax 7 October 2011.

\bibitem{IEEEhowto:three}
Knoll, Lars, \emph{Thoughts about Qt 5},\hskip 1em plus
  0.5em minus 0.4em\relax Retrieved 9 May 2011.
\
\end{thebibliography}
\end{document}
